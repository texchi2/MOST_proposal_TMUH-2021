%% Edited by Yu-Tuan Lin, Institute of Mathematics, Academia Sinica 
%% Lastest update: 2018-09-20 
%% 說明:中、英文摘要及關鍵詞 (keywords)、成果自評表(附件二)、成果彙整表(附件四)直接於
%%      科技部網站線上填寫
%%      附件三:可供推廣之研發成果資料表,應至科技部科技研發成果資訊系統(STRIKE系統)填寫
\documentclass[12pt]{article}

\usepackage[T1]{fontenc}
\usepackage{CJKutf8}

\usepackage[unicode=true,pdfusetitle,
 bookmarks=true,bookmarksnumbered=false,bookmarksopen=false,
 breaklinks=false,pdfborder={0 0 1},backref=false,colorlinks=false]
 {hyperref}

\usepackage{amsmath,amssymb}
\topmargin -10mm
\textwidth 170mm
\oddsidemargin -5mm
\evensidemargin -5mm
\textheight 220mm
%\pagestyle{empty}
\usepackage{lastpage}  
\usepackage{fancyhdr}
\pagestyle{fancy}
\usepackage{graphicx}
\usepackage{array}
\usepackage{makecell}
\usepackage{multirow}
\renewcommand{\footnotesize}{\normalsize} 
\renewcommand{\headrulewidth}{0pt}
\renewcommand{\footrulewidth}{0pt}
\lfoot{  }
\cfoot{ 共 \pageref{LastPage} 頁 第  \thepage   頁  }
\rfoot{ } 
 
\newcommand{\Z}{\mathbb{Z}}

\makeatletter
%%%%%%%%%%%%%%%%%%%%%%%%%%%%%% User specified LaTeX commands.
\CJKencfamily{UTF8}{bkai} % 使用標楷體


\AtBeginDocument{%
    \begin{CJK}{UTF8}{bkai}} % 使用標楷體
    \AtEndDocument{%
    \clearpage\end{CJK}}

\makeatother

\begin{document}
\begin{CJK}{UTF8}{}%
\vspace*{-1cm}
\noindent 附件一
\vspace*{1cm}

\begin{center}

{\bf \Large  科技部補助專題研究計畫成果報告 \\
(□ 期中進度報告/■ 期末報告)} 

\vspace{1.cm}
{\bf \Large  計畫名稱 }
\end{center}
\vspace{1.cm}

計畫類別:■個別型計畫   □整合型計畫 

計畫編號:  

執行期間:  年 月 日 至 年 月 日

\vspace{1cm}
執行機構及系所:

\vspace{1cm}
計畫主持人:

共同主持人:

計畫參與人員: 

\vspace{1.3cm}
本計畫除繳交成果報告外,另含下列出國報告,共 \underline{1} 份:

□執行國際合作與移地研究心得報告

■出席國際學術會議心得報告

□出國參訪及考察心得報告
\vspace{1.3cm}

%  期末報告處理方式:
%\begin{enumerate}
%\item 公開方式:
%  
%  □ 非列管計畫亦不具下列情形,立即公開查詢
%  
%  ■ 涉及專利或其他智慧財產權,□ 一年 ■ 二年後可公開查詢
%
%\item 「本研究」是否已有嚴重損及公共利益之發現:■ 否 ~~ □ 是
%  
%\item 「本報告」是否建議提供政府單位施政參考 ■ 否 ~~  □ 是,  
%\underline{~~~  }(請列舉提供之單位;本部不經審議,依勾選逕予轉送)
%  \end{enumerate}

\vspace{\fill}
\begin{center}
\centering{  中 \hspace{7mm} 華 \hspace{7mm}  民 \hspace{7mm}   國  \hspace{7mm} 107 \hspace{7mm} 年 \hspace{7mm} 10 \hspace{7mm} 月 \hspace{7mm} 日 } 
\end{center}

\newpage
\begin{center}
{\bf \Large 科技部補助專題研究計畫成果報告}
\vspace*{1cm}


{\bf \Large  中文計畫名稱  \\ 英文計畫名稱}

\vspace*{1cm}

{\large 計畫編號:  }

\vspace*{0.3cm}
{ 執行期限:106年08月01日至107年07月31日}

\vspace*{0.3cm}
{\large 主持人: }

\end{center}
\begin{enumerate}
 \item [一、] {\bf 前言}

 % Input Keyword
  
 \item [二、]{\bf 研究目的}
% 輸入 研究目的 

 \item [三、]{\bf 文獻探討}
% 輸入 研究方法 

 \item [四、]{\bf 研究方法}
% 輸入 結果與討論

\item[五、] {\bf 結果與討論}
\end{enumerate}

\newpage
\vspace*{-1cm}
\noindent 附件二
\vspace*{1cm}
\begin{center}
{\bf \Large 科技部補助專題研究計畫成果自評表}
\end{center}

\vspace*{1cm}

請就研究內容與原計畫相符程度、達成預期目標情況、研究成果之學術或應用價值(簡要敘述成果所代表之意義、價值、影響或進一步發展之可能性)、是否適合在學術期刊發表或申請專利、主要發現(簡要敘述成果是否具有政策應用參考價值及具影響公共利益之重大發現)或其他有關價值等,作一綜合評估。

\begin{enumerate}
\item 請就研究內容與原計畫相符程度、達成預期目標情況作一綜合評估 \\
■ 達成目標 \\
□ 未達成目標(請說明,以100字為限) \\
\hspace*{2cm } □ 實驗失敗 \\
\hspace*{2cm } □ 因故實驗中斷 \\
\hspace*{2cm } □ 其他原因 \\
說明:

\vspace{1cm}

\item 研究成果在學術期刊發表或申請專利等情形(請於其他欄註明專利及技轉之證號、合約、申請及洽談等詳細資訊) \\
論文:■ 已發表 □ 未發表之文稿 ■ 撰寫中 □ 無 \\
專利:□ 已獲得 □ 申請中 ■ 無 \\
技轉:□ 已技轉 □ 洽談中 ■ 無 \\
其他:(以200字為限) \\

\vspace{1cm}

\item 請依學術成就、技術創新、社會影響等方面,評估研究成果之學術或應用價值(簡要敘述成果所代表之意義、價值、影響或進一步發展之可能性,以500字為限)。

\vspace{1cm}
\item 本研究具有政策應用參考價值:  □否    □是,建議提供機關\underline{\, \, \, \,}  \newline
    (勾選「是」者,請列舉建議可提供施政參考之業務主管機關) \newline
   本研究具影響公共利益之重大發現:□否    □是  \newline
   說明:(以150字為限)

\end{enumerate}


\newpage
\vspace*{-1cm}
\noindent 附件五
\begin{center}
{\bf \Large 科技部補助專題研究計畫執行國際合作與移地研究心得報告}
\end{center}

\hfill{\small 日期:   年   月   日 }
\begin{table}[h!]{\renewcommand{\arraystretch}{2}
    \begin{tabular}{|c|l|c|l|}
    \hline
   ~ 計畫編號 ~ & \multicolumn{3}{l|}{MOST  --} \\ \hline
   ~ 計畫名稱 ~ & \multicolumn{3}{l|}{} \\ \hline
    \makecell{~ 出國人員 ~\\ 姓名} & ~ \hspace*{5cm} & \makecell{~ 服務機關 ~\\及職稱} & ~ \hspace*{5cm} \\ \hline
   ~ 出國時間 ~ & \makecell[l]{~年~月~日至\\  ~年~月~日} & 出國地點 & ~ \\ \hline
    \makecell{~ 出國研究 ~\\目的} & \multicolumn{3}{l|}{□實驗 □田野調查 □採集樣本 □國際合作研究 □使用國外研究設施} \\ \hline
    \end{tabular}
}
\end{table}
\begin{enumerate}
\item[一、] 執行國際合作與移地研究過程 

\item[二、]研究成果 

\item[三、]建議 

\item[四、]本次出國若屬國際合作研究,雙方合作性質係屬:(可複選) \\
□分工收集研究資料 \\
□交換分析實驗或調查結果 \\
□共同執行理論建立模式並驗証 \\
□共同執行歸納與比較分析 \\
□元件或產品分工研發 \\
□其他 (請填寫) 

\item[五、]其他

\end{enumerate}

\newpage
\vspace*{-1cm}
\noindent 附件六
\begin{center}
{\bf \Large 科技部補助專題研究計畫出席國際學術會議心得報告}
\end{center}

\hfill{\small 日期:   年   月   日 }
\begin{table}[h!]{\renewcommand{\arraystretch}{2}
    \begin{tabular}{|l|l|l|l|}
    \hline
   ~ 計畫編號 ~ & \multicolumn{3}{l|}{MOST  --} \\ \hline
   ~ 計畫名稱 ~ & \multicolumn{3}{l|}{} \\ \hline
    \makecell{~ 出國人員 ~\\ 姓名} & ~ \hspace*{5cm} & \makecell{~ 服務機關 ~\\及職稱} & ~ \hspace*{5cm} \\ \hline    
   ~ 會議時間 ~ & \makecell[l]{~年~月~日至\\  ~年~月~日} & 會議地點 & ~ \\ \hline
   ~ 會議名稱 ~ & \multicolumn{3}{l|}{ \makecell[l]{(中文)  \\ (英文) }} \\ \hline
     ~ 發表題目 ~ & \multicolumn{3}{l|}{ \makecell[l]{(中文)  \\ (英文)  }} \\ \hline
    \end{tabular}
}
\end{table}
\begin{enumerate}
\item[一、] 參加會議經過

\item[二、] 與會心得

\item[三、] 發表論文全文或摘要

\item[四、] 建議

\item[五、] 攜回資料名稱及內容

\item[六、] 其他

\end{enumerate}

\newpage
\vspace*{-1cm}
\noindent 附件七
\begin{center}
{\bf \Large 科技部補助專題研究計畫執行出國參訪及考察心得報告}
\end{center}

\hfill{\small 日期:   年   月   日 }
\begin{table}[h!]{\renewcommand{\arraystretch}{2}
    \begin{tabular}{|l|l|l|l|}
    \hline
   ~ 計畫編號 ~ & \multicolumn{3}{l|}{MOST  --} \\ \hline
   ~ 計畫名稱 ~ & \multicolumn{3}{l|}{} \\ \hline
    \makecell{~ 出國人員 ~\\ 姓名} & ~ \hspace*{5cm} & \makecell{~ 服務機關 ~\\及職稱} & ~ \hspace*{5cm} \\ \hline    
   ~ 出國時間 ~ & \makecell{~年~月~日至\\  ~年~月~日} & 出國地點 & ~ \\ \hline
  
    \end{tabular}
}
\end{table}
\begin{enumerate}
\item[一、] 參訪及考察過程

\item[二、] 心得

\item[三、] 建議

\item[四、]  其他

\end{enumerate}







\end{CJK}

\end{document}
